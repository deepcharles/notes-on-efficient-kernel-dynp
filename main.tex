\title{Pseudo-code for the efficient kernel change point detection}
\author{
        Charles Truong
}
\date{}

\documentclass[12pt]{article}

\usepackage{amsmath}
\usepackage{amssymb}
\usepackage{algorithm}
\usepackage{algorithmic} 
\usepackage{bm}
\usepackage{mathtools}

\newcommand\maxop{{\max}_\Omega}
\newcommand\vecc{\text{vec}\,}
\newcommand\RR{\mathbb{R}}
\newcommand\NN{\mathbb{N}}
\newcommand\HH{\mathcal{H}}
\newcommand{\norm}[1]{\left\lVert#1\right\rVert}
\newcommand{\sca}[2]{\left\langle#1\mid #2\right\rangle}
\newcommand{\pluseq}{\mathrel{+}=}
\newcommand{\TT}{\mathcal{T}}
\newcommand{\TThat}{\widehat{\mathcal{T}}}
\DeclareMathOperator*{\argmin}{arg\,min}
\newcommand{\sslash}{\mathbin{/\mkern-6mu/}}
\DeclarePairedDelimiter\ceil{\lceil}{\rceil}
\DeclarePairedDelimiter\floor{\lfloor}{\rfloor}

\begin{document}
\maketitle

\section{Introduction}
In the following, $y=\{y_t\}_{t=0}^{T-1}$ denotes a $\RR^d$-valued signal with $T$ samples.
The original signal $y$ is mapped onto a reproducing Hilbert space (rkhs) $\HH$ associated with a user-defined kernel function $k(\cdot, \cdot):\RR^d\times\RR^d \leftarrow \RR$.
The mapping function $\phi:\RR^d \leftarrow\HH$ onto this rkhs is implicitly defined by $\phi(y_t)=k(y_t, \cdot)\in\HH$, resulting in the following inner-product and norm:
\begin{equation}
    \sca{\phi(y_s)}{\phi(y_t)}_\HH = k(y_s, y_t) \quad\text{and}\quad \norm{\phi(y_t)}_\HH = k(y_t, y_t)
\end{equation}
for any samples $y_s, y_t\in\RR^d$.

\paragraph{Notations.}
For two indexes $a$ and $b$ ($a<b$), $y_{a..b} = \{y_a,y_{a+1},\dots,y_{b-1}\}$ is the sub-signal between $a$ (included) and $b$ (excluded).
(For instance, $y_{0..T}$ is the entire signal.)
A set of $K$ change points is denoted $\TT=\{t_1,t_2,\dots,t_K\}$ and its cardinal is $|\TT|$.
By convention, $t_0=0$ and $t_{|\TT|+1}=T$.
Denote by $\sslash $ the integer division, i.e.\ $a\sslash b= \floor{a/b}$ where $\floor{\cdot}$ is the floor function.

\section{Related work}

\cite{Celisse2017}

\section{Problem formulation}
For this segmentation task, the cost function $c(\cdot)$ is given by
\begin{equation}
    c(y_{a..b}) := \sum_{t=a}^{b-1} \norm{\phi(y_t) - \mu_{a..b}}_{\HH}^2 = \sum_{t=a}^{b-1} \norm{\phi(y_t)}_{\HH}^2 - \frac{1}{b-a}\norm{\sum_{t=a}^{b-1} \phi(y_t)}^2_\HH
\end{equation}
where $\mu_{a..b} = (\sum_{t=a}^{b-1} \phi(y_t))/(b-a)$.
It is easy to see that $c(\cdot)$ can be rewritten using only the kernel $k(\cdot, \cdot)$:
\begin{equation}
    c(y_{a..b}) = \sum_{t=a}^{b-1} k(y_t, y_t) - \frac{1}{b-a} \sum_{s,t=a}^{b-1} k(y_s,y_t).
\end{equation}
The best segmentation (i.e.\ the best set of change points) with $K$ ($K>1$) breaks is denoted $\TThat_K$, defined by
\begin{equation}
    \TThat_K := \argmin_{\TT, |\TT|=K} V(y_{0..T}, \TT)\quad\text{where}\quad V(y_{0..T}, \TT):= \sum_{k=0}^{K} c(y_{t_k..t_{k+1}}).
    \label{eq:min-sum-of-cost}
\end{equation}
Let $V_K(y_{0..T})$ be the minimum sum of costs for $K$ changes points:
\begin{equation}
    V_K(y_{0..T}) := V(y_{0..T}, \TThat_K) \quad\text{for } K\geq1.
\end{equation}
By convention,
\begin{equation}
    V_0(y_{a..b}) := c(y_{a..b}).
\end{equation}
The following equality is at the heart of the dynamic programming formulation:
\begin{equation}
    V_K(y_{0..T}) = \min_{t=K,..,T-1} \left[V_{K-1}(y_{0..t}) + c(y_{t..T})\right].
\end{equation}

\section{Pseudo-code}

\paragraph{Rewrite the cost.}
Rewrite the cost function using two easily computable quantities:
\begin{equation}
    c(y_{a..b}) = D_{a..b} - \frac{1}{b-a} S_{a..b, a..b}
\end{equation}
where
\begin{equation}
    D_{a..b} := \sum_{t=a}^{b-1} k(y_t, y_t) \quad\text{and}\quad S_{a..b, a'..b'} := \sum_{a\leq s <b} \sum_{a'\leq t <b'} k(y_s, y_t).
\end{equation}
(Note that $D_{a..a}=0$ and $S_{a..a, a'..b'}=S_{a..b, a'..a'}=0$.)
It is easy to see that
\begin{equation}
    S_{a..b, a..b} = S_{0..a, 0..a} + S_{0..b, 0..b} - 2S_{0..a, 0..b}.
\end{equation}
Both $D_{a..b}$ and $S_{a..b, a..b}$ can be computed iteratively, thanks to the following identities:
\begin{equation}
    S_{0..a, 0..b+1} = S_{0..a, 0..b} + S_{0..a, b..b+1}\quad\text{and}\quad S_{0..a+1, b..b+1} = S_{0..a, b..b+1} + k(x_a, x_b)
\end{equation}
as well as
\begin{equation}
    D_{a..b} = D_{0..b} - D_{0..a}\quad\text{and}\quad D_{0..b+1} = D_{0..b} + k(y_b, y_b).
\end{equation}

\paragraph{The path matrix.}
In Algorithm~\ref{alg:dynp}, the output is the path matrix $M_{\text{path}}$ which is such that $M_{\text{path}}[t, k]$ is the last change point of the optimal segmentation with $k$ breaks of $y_{0..t}$.
To recover the list of changes of $y$, for a given number $K$ of change points, one needs to iterate to return
\begin{equation*}
    t_K = M_{\text{path}}[T, K], t_{K-1}=M_{\text{path}}[t_K, K-1], t_{K-2}=M_{\text{path}}[t_{K-1}, K-2], \dots
\end{equation*}

\paragraph{Jump and minimum segment size.}
In order to decrease the complexity, the change point detection can be restricted to a subset of indexes.
Denote the jump by $\delta$ and the minimum segment size by $\Delta$.
The minimization~\eqref{eq:min-sum-of-cost} is done over the admissible segmentations $\TT=\{t_1,t_2,\dots,t_K\}$ such that
\begin{equation}
    t_k \text{ is a multiple of }\delta\text{ and } t_{k+1}-t_k \geq \Delta.
\end{equation}


\begin{algorithm}
    \small
    \caption{Compute the best segmentation with $K$ change points}
    \label{alg:dynp} % and a label for \ref{} commands later in the document
    \begin{algorithmic} % enter the algorithmic environment
        \REQUIRE Signal $y = \{y_t\}_t$ with $T$ samples, number $K$ of change points
        \STATE \COMMENT{Initialize arrays}
        \STATE $D \gets \bm{0}\in\mathbb{R}^{T+1}$ \COMMENT{sub-sums $D_{0..t}$}
        \STATE $S \gets \bm{0}\in\mathbb{R}^{T+1}$ \COMMENT{sub-sums $S_{s..t}$}
        \STATE $M_V \gets \bm{0}\in\mathbb{R}^{(T+1)\times (K+1)}$ \COMMENT{cost matrix, to hold sums of costs}
        \STATE $M_{\text{path}} \gets \bm{0}\in\mathbb{R}^{(T+1)\times (K+1)}$ \COMMENT{path matrix, to hold the change point indexes}
        \STATE $c_{\text{current}}, v_{\text{current}}, r_{\text{current}}\in\RR$ \COMMENT{current cost, sum of costs, and $R_{s..t}$}
        % -----------------------------------------------------------------------------------------
        \STATE
        \FORALL{$t=1,\dots,T$}
        \STATE \COMMENT{handle $y_{0..t}=\{y_0,\dots,y_{t-1}\}$}
        \STATE
        \STATE \COMMENT{Compute $D[t]$}
        \STATE $D[t] \gets D[t-1] + k(y_{t-1}, y_{t-1})$ \COMMENT{$D[t]=D_{0..t}$}
        \STATE
        \STATE \COMMENT{Compute $S[t-1],S[t-2],\dots,S[0]$}
        \STATE $r_{\text{current}}\gets0$
        \FORALL{$s=t-1,\dots,0$}
        \STATE $r_{\text{current}} \gets r_{\text{current}} + k(y_{s}, y_{t-1})$ \COMMENT{$r_{\text{current}}=R_{s..t}$}
        \STATE $S[s] \gets S[s] - k(y_{t-1}, y_{t-1}) + 2\times r_{\text{current}}$ \COMMENT{$S[s]=S_{s..t}$}
        \ENDFOR
        \STATE
        \STATE \COMMENT{Compute segmentations}
        \STATE $M_{V}[t, 0] \gets D[t] - S[0]/t$ \COMMENT{Recall that $M_{V}[t, 0]=V_0(y_{0..t})=c(y_{0..t})$}
        \FORALL{$s=1,\dots,t-1$}
        \STATE $c_{\text{current}} \gets D[t]-D[s] - S[s]/(t-s)$ \COMMENT{$=c(y_{s..t})$}
        \FORALL{$k=1,\dots,\min(K, s)$}
        \STATE $v_{\text{current}} \gets M_V[s, k-1] + c_{\text{current}}$ \COMMENT{$=V_{k-1}(y_{0..s})+c(y_{s..t})$}
        \IF{$s=k$}
        \STATE $M_V[t, k] \gets v_{\text{current}}$
        \STATE $M_{\text{path}}[t, k] \gets s$
        \ELSE
        \IF{$M_V[t, k]>v_{\text{current}}$}
        \STATE $M_V[t, k] \gets v_{\text{current}}$
        \STATE $M_{\text{path}}[t, k] \gets s$
        \ENDIF
        \ENDIF
        \ENDFOR
        \ENDFOR
        \ENDFOR
        \RETURN $M_{\text{path}}$
    \end{algorithmic}
\end{algorithm}


\begin{algorithm}
    \scriptsize
    \caption{Compute the best segmentation with $K$ change points (with jump and minimum segment size)}
    \label{alg:dynp_jump_min_size} % and a label for \ref{} commands later in the document
    \begin{algorithmic} % enter the algorithmic environment
        \REQUIRE Signal $y = \{y_t\}_t$ with $T$ samples, number $K$ of change points, jump $\delta$, minimum segment size $\Delta$
        \STATE \COMMENT{Initialize}
        \STATE $q\gets \ceil{T/\delta}$ \COMMENT{ceil function}
        \STATE $D \gets \bm{0}\in\mathbb{R}^{q+1}$ \COMMENT{sub-sums $D_{0..t}$}
        \STATE $S_{\text{off-diag}} \gets \bm{0}\in\mathbb{R}^{q+1}$ \COMMENT{sub-sums $S_{0..s, 0..t}$}
        \STATE $S_{\text{diag}} \gets \bm{0}\in\mathbb{R}^{q+1}$ \COMMENT{sub-sums $S_{0..t, 0..t}$}
        \STATE $M_V \gets \bm{0}\in\mathbb{R}^{(q+1)\times (K+1)}$ \COMMENT{cost matrix, to hold sums of costs}
        \STATE $M_{\text{path}} \gets \bm{0}\in\mathbb{R}^{(q+1)\times (K+1)}$ \COMMENT{path matrix, to hold the change point indexes}
        \STATE $d_{\text{current}},s_{\text{current}}\gets0$ \COMMENT{accumulator variables for $D_{0..t}$ and $S_{0..s, 0..t}$}
        \STATE $c_{\text{current}}, v_{\text{current}}\in\RR$ \COMMENT{current cost, sum of costs}
        % -----------------------------------------------------------------------------------------
        \STATE
        \FORALL{$q_t=1,\dots,q$}
        \STATE $S_{\text{off-diag}}[q_t]\gets S_{\text{off-diag}}[q_t-1]$
        \FORALL{$t=(q_t-1)\delta+1,\dots,\min(q_t\delta,T)$}
        \STATE $d_{\text{current}}\gets d_{\text{current}} + k(y_{t-1}, y_{t-1})$
        \STATE \COMMENT{Update $S_{\text{off-diag}}[1],S_{\text{off-diag}}[2],\dots,S_{\text{off-diag}}[q_t]$}
        \STATE $acc\gets0$
        \FORALL{$q_s=1,\dots,q_t-1$}
        \FORALL{$s=(q_s-1)\delta+1,\dots,q_s\delta$}
        \STATE $acc \gets acc + k(y_{s-1}, y_{t-1})$
        \ENDFOR
        \STATE $S_{\text{off-diag}}[q_s] \gets S_{\text{off-diag}}[q_s] + acc$ \COMMENT{$S_{\text{off-diag}}[q_s]=S_{0..q_s\delta, 0..q_t\delta}$}
        \ENDFOR
        \FORALL{$s=(q_t-1)\delta+1,\dots t-1$}
        \STATE $acc \gets acc + k(y_{s-1}, y_{t-1})$
        \ENDFOR
        \STATE $S_{\text{off-diag}}[q_t] \gets S_{\text{off-diag}}[q_t] + 2\times acc +  k(y_{t-1}, y_{t-1})$
        \ENDFOR
        \STATE $D[q_t] \gets d_{\text{current}}$ \COMMENT{$D[q_t]=D_{0..q_t\delta}$}
        \STATE $S_{\text{diag}}[q_t] \gets S_{\text{off-diag}}[q_t]$
        \STATE
        \STATE \COMMENT{Compute segmentations}
        \STATE $M_{V}[q_t, 0] \gets D[q_t] - S_{\text{diag}}[q_t]/(q_t\delta)$ \COMMENT{Recall that $M_{V}[q_t, 0]=c(y_{0..q_t\delta})$}
        \FORALL{$q_s=1,\dots,q_t-\floor{(\Delta-1)/\delta}-1$}
        \STATE $c_{\text{current}} \gets D[q_t]-D[q_s] - (S_{\text{diag}}[q_t]+S_{\text{diag}}[q_s]-2S_{\text{off-diag}}[q_s])/(q_t\delta-q_s\delta)$ \COMMENT{$=c(y_{s..t})$}
        \FORALL{$k=1,\dots,\min(K, q_s)$}
        \STATE $v_{\text{current}} \gets M_V[q_s, k-1] + c_{\text{current}}$ \COMMENT{$=V_{k-1}(y_{0..q_s\delta})+c(y_{q_s\delta..q_t\delta})$}
        \IF{$q_s=k$}
        \STATE $M_V[q_t, k] \gets v_{\text{current}}$
        \STATE $M_{\text{path}}[q_t, k] \gets q_s$
        \ELSE
        \IF{$M_V[q_t, k]>v_{\text{current}}$}
        \STATE $M_V[q_t, k] \gets v_{\text{current}}$
        \STATE $M_{\text{path}}[q_t, k] \gets q_s$
        \ENDIF
        \ENDIF
        \ENDFOR
        \ENDFOR
        \ENDFOR
        \RETURN $M_{\text{path}}$
    \end{algorithmic}
\end{algorithm}


\begin{equation}
    S_{s..t} = S_{s..t-1} - k(y_{t-1}, y_{t-1}) + 2\sum_{u=s}^{t-1} k(y_{t-1}, y_u)
\end{equation}
\begin{algorithm}
    \small
    \caption{Compute the best segmentation with $K$ change points}
    \label{alg:dynp_min_size} % and a label for \ref{} commands later in the document
    \begin{algorithmic} % enter the algorithmic environment
        \REQUIRE Signal $y = \{y_t\}_t$ with $T$ samples, number $K$ of change points, minimum segment size $\Delta$
        \STATE \COMMENT{Initialize arrays}
        \STATE $D \gets \bm{0}\in\mathbb{R}^{T+1}$ \COMMENT{sub-sums $D_{0..t}$}
        \STATE $S \gets \bm{0}\in\mathbb{R}^{T+1}$ \COMMENT{sub-sums $S_{s..t}$}
        \STATE $M_V \gets \bm{0}\in\mathbb{R}^{(T+1)\times (K+1)}$ \COMMENT{cost matrix, to hold sums of costs}
        \STATE $M_{\text{path}} \gets \bm{0}\in\mathbb{R}^{(T+1)\times (K+1)}$ \COMMENT{path matrix, to hold the change point indexes}
        \STATE $c_{\text{current}}, v_{\text{current}}, r_{\text{current}}\in\RR$ \COMMENT{current cost, sum of costs, and $R_{s..t}$}
        % -----------------------------------------------------------------------------------------
        \STATE
        \FORALL{$t=1,\dots,T$}
        \STATE \COMMENT{handle $y_{0..t}=\{y_0,\dots,y_{t-1}\}$}
        \STATE
        \STATE \COMMENT{Compute $D[t]$}
        \STATE $D[t] \gets D[t-1] + k(y_{t-1}, y_{t-1})$ \COMMENT{$D[t]=D_{0..t}$}
        \STATE
        \STATE \COMMENT{Compute $S[t-1],S[t-2],\dots,S[0]$}
        \STATE $r_{\text{current}}\gets0$
        \FORALL{$s=t-1,\dots,0$}
        \STATE $r_{\text{current}} \gets r_{\text{current}} + k(y_{s}, y_{t-1})$ \COMMENT{$r_{\text{current}}=R_{s..t}$}
        \STATE $S[s] \gets S[s] + 2\times r_{\text{current}} - k(y_{t-1}, y_{t-1}) $ \COMMENT{$S[s]=S_{s..t}$}
        \ENDFOR
        \STATE
        \STATE \COMMENT{Compute segmentations}
        \STATE $M_{V}[t, 0] \gets D[t] - S[0]/t$ \COMMENT{Recall that $M_{V}[t, 0]=V_0(y_{0..t})=c(y_{0..t})$}
        \FORALL{$s=\Delta,\dots,t-\Delta$}
        \STATE $c_{\text{current}} \gets D[t]-D[s] - S[s]/(t-s)$ \COMMENT{$=c(y_{s..t})$}
        \FORALL{$k=1,\dots,\min(K, \floor{s/\Delta})$}
        \STATE $v_{\text{current}} \gets M_V[s, k-1] + c_{\text{current}}$ \COMMENT{$=V_{k-1}(y_{0..s})+c(y_{s..t})$}
        \IF{$s=k\Delta$}
        \STATE $M_V[t, k] \gets v_{\text{current}}$
        \STATE $M_{\text{path}}[t, k] \gets s$
        \ELSE
        \IF{$M_V[t, k]>v_{\text{current}}$}
        \STATE $M_V[t, k] \gets v_{\text{current}}$
        \STATE $M_{\text{path}}[t, k] \gets s$
        \ENDIF
        \ENDIF
        \ENDFOR
        \ENDFOR
        \ENDFOR
        \RETURN $M_{\text{path}}$
    \end{algorithmic}
\end{algorithm}

\begin{algorithm}
    \footnotesize
    \caption{Compute the best penalized segmentation}
    \label{alg:pelt_min_size} % and a label for \ref{} commands later in the document
    \begin{algorithmic} % enter the algorithmic environment
        \REQUIRE Signal $y = \{y_t\}_t$ with $T$ samples, linear smoothing term $\beta$, minimum segment size $\Delta$
        \STATE \COMMENT{Initialize arrays}
        \STATE $D \gets \bm{0}\in\mathbb{R}^{T+1}$ \COMMENT{sub-sums $D_{0..t}$}
        \STATE $S \gets \bm{0}\in\mathbb{R}^{T+1}$ \COMMENT{sub-sums $S_{s..t}$}
        \STATE $M_V \gets \bm{0}\in\mathbb{R}^{(T+1)}$ \COMMENT{cost matrix, to hold sums of costs}
        \STATE $M_{pruning} \gets \bm{0}\in\mathbb{R}^{(T+1)}$ \COMMENT{used for the pruning rule}
        \STATE $M_{\text{path}} \gets \bm{0}\in\mathbb{R}^{(T+1)}$ \COMMENT{path matrix, to hold the change point indexes}
        \STATE $c_{\text{current}}, v_{\text{current}}, r_{\text{current}}\in\RR$ \COMMENT{current cost, sum of costs, and $R_{s..t}$}
        \STATE $s_{\text{min}}\gets \Delta$ \COMMENT{leftmost admissible change points}
        % -----------------------------------------------------------------------------------------
        \STATE
        \STATE \COMMENT{handle $y_{0..t}=\{y_0,\dots,y_{t-1}\}$}
        \FORALL{$t=1,\dots,T$}
        \STATE $D[t] \gets D[t-1] + k(y_{t-1}, y_{t-1})$ \COMMENT{$D[t]=D_{0..t}$}
        \STATE \COMMENT{Compute $S[t-1],S[t-2],\dots,S[s_{\text{min}}]$}
        \STATE $r_{\text{current}}\gets0$
        \FORALL{$s=t-1,\dots,s_{\text{min}}$}
        \STATE $r_{\text{current}} \gets r_{\text{current}} + k(y_{s}, y_{t-1})$ \COMMENT{$r_{\text{current}}=R_{s..t}$}
        \STATE $S[s] \gets S[s] + 2\times r_{\text{current}} - k(y_{t-1}, y_{t-1}) $ \COMMENT{$S[s]=S_{s..t}$}
        \ENDFOR
        \STATE \COMMENT{Compute segmentations}
        \STATE $s\gets s_{\text{min}} $
        \STATE $c_{\text{current}} \gets D[t]-D[s] - S[s]/(t-s)$ \COMMENT{$=c(y_{s..t})$}
        \STATE $v_{\text{current}} \gets M_V[s] + c_{\text{current}} + \beta$ \COMMENT{$=V(y_{0..s})+c(y_{s..t})+\beta$}
        \STATE $M_{\text{pruning}}[s] \gets M_V[s] + c_{\text{current}}$ \COMMENT{$=V(y_{0..s})+c(y_{s..t})$}
        \STATE $M_V[t] \gets v_{\text{current}}$
        \STATE $M_{\text{path}}[t] \gets s$
        \FORALL{$s=s_{\text{min}}+1,\dots,t-\Delta$}
        \STATE $c_{\text{current}} \gets D[t]-D[s] - S[s]/(t-s)$ \COMMENT{$=c(y_{s..t})$}
        \STATE $v_{\text{current}} \gets M_V[s] + c_{\text{current}} + \beta$ \COMMENT{$=V(y_{0..s})+c(y_{s..t})+\beta$}
        \STATE $M_{\text{pruning}}[s] \gets M_V[s] + c_{\text{current}}$
        \IF{$M_V[t]>v_{\text{current}}$}
        \STATE $M_V[t] \gets v_{\text{current}}$
        \STATE $M_{\text{path}}[t] \gets s$
        \ENDIF
        \ENDFOR
        \STATE \COMMENT{Pruning}
        \WHILE{$M_{\text{pruning}}[s_{\text{min}}] \geq M_V[t]$ and $s_{\text{min}}\leq t-\Delta$}
        \STATE $s_{\text{min}}\gets s_{\text{min}} + 1$
        \ENDWHILE
        \ENDFOR
        \RETURN $M_{\text{path}}$
    \end{algorithmic}
\end{algorithm}



\bibliographystyle{abbrv}
\bibliography{biblio}

\end{document}